\chapter{PerOpteryx Extension}
\label{ch:PerOpt}

PerOpteryx, briefly introduced in \autoref{sec:Foundations:peropteryx}, is a model optimization framework. It is designed to calculate performance and cost optimised PCM models. For this purpose PerOpetryx uses an evolutionary algorithm to generate new PCM candidates. PerOpteryx uses a \textit{Design Decisions} EMF model to create a \textit{Design Space}. The design space is defined via \textit{Degrees Of Freedom}. Every degree of freedom allows for a finite amount of \textit{Decisions}. A \textit{Candidate} consists of one choice per degree of freedom and represents a PCM instance. 

Every candidate needs to be evaluated in order to decide if the \textit{Decisions} made during its constructions lead to a good results. Each evaluator can produce multiple results per analysis run. Each result belongs to a certain \textit{QML dimension}. A dimension has \textit{Objectives} and/or \textit{Constraints}, which helps the evolutionary algorithm to find the Pareto-optimal candidates. Every evaluator is encapsulated in an \textit{Eclipse Plug-in} \cite{PerOpteryx.b}. Since we need another evaluator, we need to create a new plug-in.

\section{Plug-in Design}
\label{sec:PerOpt:design}

We want to provide a \textit{Privacy Analysis} evaluator for PerOpteryx, while using our previously developed \textit{Privacy Anylsis} (\autoref{ch:PrivacyAnalysis}). Since both systems are based on the Palladio Component Model, the privacy analysis itself can be used as described in \autoref{ch:PrivacyAnalysis}. However, PerOpteryx does not know a "Privacy Dimension", which is required, since every analysis result needs an according dimension. A new dimension is required, since using a pre-existent dimension - like the "cost dimension" - would undermine the evolutionary algorithms optimizations effort. The privacy analysis has a single result, which is a \textit{Constraint}. It states, that no privacy violation is permitted.

As mentioned before, we need to create a new \textit{QML Dimension}, the \textit{Privacy Dimension}, which is referenced in a \textit{QML Contract Type}. The contract type references all the evaluation dimensions. The \textit{QML Declaration} references the contract type and specifies the actual objectives and constraints for the dimensions. Further, it specifies a \textit{QML Profile}, which \textit{Usage Model} is used for the evaluation.


\section{PerOpteryx Modification}

PerOpteryx' prior structure considers every generated candidate to be valid, only with different runtime results. This is incorrect, when the privacy dimension is included. Only privacy compliant models are valid options. As a result, we can abort the evaluation of the current PCM model, if the privacy constraint is broken. However, this means we need to execute the privacy evaluation first, check if the constraint is broken, break the evaluation if the model is invalid and fill all other objectives and constraints with according values.

As mentioned above, every evaluation is encapsulated in an \textit{Eclipse Plug-in}. So, every evaluation is represented as \textit{Proxy Analysis} in PerOpteryx, who has no information on what evaluation he is actually currently executing. To save evaluation time and increase our search space, we need to execute the Privacy Analysis first, while not breaking the generic evaluation characteristic. As a result, the \textit{IAnalysis} interface for every evaluation was extended with a \textit{Evaluation Complexity} query, returning an Enum representing the analysis runtime duration. PerOpteryx was modified in a way, that analysis returning the value \textit{VERY\_SHORT} get executed first. The Privacy Analysis evaluator returns this value.

Further, PerOpteryx was modified, to output the \textit{most cost efficient} candidate once all evaluation iterations are completely executed. The cost criteria was chosen over performance due to PerOpteryx's tendency to spread the allocation over many server to optimize the performance. The cost optimal model tends to group components on servers, reducing the servers required and therefore saving money, while still having a decent performance.


 

