\chapter{PerOpteryx Extension}
\label{ch:PerOpt}

PerOpteryx, briefly introduced in \autoref{sec:Foundations:peropteryx}, is a model optimization framework. It is designed to calculate performance and cost optimised PCM models. For this purpose PerOpetryx uses an evolutionary algorithm to generate new PCM candidates. Every candidate needs to be evaluated in order to decide if the decisions made during its constructions lead to a good results. Each evaluator can produce multiple results per analysis runs. Each result belongs to a certain QML dimension. A dimension can have \textit{Objectives} and/or \textit{Constraints}, which help the evolutionary algorithm to find the pareto-optimal candidates. Every evaluator is encapsulated in an \textit{Eclipse PlugIn}. Since we need another evaluator, we need to create new plug-in.

\section{PlugIn Design}
\label{sec:PerOpt:design}

We want to provide a \textit{Privacy Analysis} evaluator for PerOpteryx, while using our previously developed \textit{Privacy Anylsis} (\autoref{ch:PrivacyAnalysis}). Since both systems are based on the Palladio Component Model, the privacy analysis itself can be used as described in \autoref{ch:PrivacyAnalysis}. However, PerOpteryx doesn't know a "Privacy Dimension", which is required, since every analysis result needs an according dimension. A new dimension is required, since using a pre-existent dimension - like the "cost dimension" - would undermine the evolutionary algorithms optimizations effort. The privacy analysis has a single result, which is a \textit{Constraint}. It states, that no privacy violation is permitted.

As mentioned before, we need to create a new \textit{QML Dimension}, the \textit{Privacy Dimension}, which is referenced in a \textit{QML Contract Type}. The contract type references all the evaluation dimensions. The \textit{QML Declaration} references the contract type and specifies the actual objectives and constraints for the dimensions. Further it specifies a \textit{QML Profile}, which \textit{Usage Model} is used for the evaluation.


\section{PerOpteryx Modification}

PerOpteryx' prior structure considers every generated candidate to be valid, only with different runtime results. This is incorrect, when the privacy dimension is included. Only privacy compliant models are valid. 
 

