%% LaTeX2e class for student theses
%% sections/abstract_de.tex
%% 
%% Karlsruhe Institute of Technology
%% Institute for Program Structures and Data Organization
%% Chair for Software Design and Quality (SDQ)
%%
%% Dr.-Ing. Erik Burger
%% burger@kit.edu
%%
%% Version 1.3, 2016-12-29

\Abstract

Mit der ständig wachsenden Zahl von verteilten Cloudanwendungen und immer mehr Datenschutzverordnungen wächst das Interesse an legalen Cloudanwendungen. Jedoch ist vielen Betreibern der Legalitätsstatus ihrer Anwendung nicht bekannt. In 2018 wird die neue EU Datenschutzverordnung in Kraft treten. Diese Verordnung beinhaltet empfindliche Strafen für Datenschutzverletzungen. Einer der wichtigsten Faktoren für die Einhaltung der Datenschutzverordnung ist die Verarbeitung von Stammdaten von EU-Bürgern innerhalb der EU. Wir haben für diese Regelung eine Privacy Analyse entwickelt, formalisiert, implementiert und evaluiert. Außerdem haben wir mit iObserve Privacy ein System nach dem MAPE Prinzip entwickelt, dass automatisch Datenschutzverletzungen erkennt und eine alternatives, datenschutzkonformes Systemhosting errechnet. Zudem migriert iObserve Privacy die Cloudanwendung entsprechend dem alternativen Hosting automatisch. Wir eine rechtskonforme Verteilung der Cloudanwendung gewährleisten, ohne das System in seiner tiefe zu analysieren oder zu verstehen. Jedoch benötigen wir die Closed World Assumption. Wir benutzen PerOpteryx für die Generierung von rechtskonformen, alternativen Hostings. Basierend auf diesem Hosting errechnen wir eine Sequenz von Adaptionsschritten zur Wiedererlangung der Rechtskonformität.  Wenn Fehler auftreten nutzen wir das Operator-in-the-loop Prinzip von iObserve. Als Datengrundlage nutzen wir das Palladio Component Model. Diese Thesis beschreiben wir detailliert die Konzepte, weisen auf Implementierungsdetails hin und evaluieren iObserve nach Präzision und Skalierbarkeit.

