%% LaTeX2e class for student theses
%% sections/content.tex
%% 
%% Karlsruhe Institute of Technology
%% Institute for Program Structures and Data Organization
%% Chair for Software Design and Quality (SDQ)
%%
%% Dr.-Ing. Erik Burger
%% burger@kit.edu
%%
%% Version 1.3, 2016-12-29

\chapter{Overview}
\label{ch:Overview}

In this chapter we will give an overview on the approach developed during this thesis and the according research. The system is massively extending iObserve, while keeping its original purpose. For the fundamentals on iObserve see \autoref{sec:Foundations:iobserve}. All extensions are made for accomplishing the goals, defined in \autoref{sec:Introduction:goals}. The extended iObserve is referred to as \textit{iObserve Privacy} during this thesis. iObserve Privacys architecture is a filter pipeline, where each filter represents one stage of the MAPE feedback loop (\autoref{sec:Foundations:mape}).


\begin{figure}[h]
	\centering
	\includegraphics[width=0.99\textwidth]{pictures/pipeline}
	\caption{iObserve Privacy pipeline}
	\label{fig:pipeline}
\end{figure}

The initial step, monitoring, is provided by the original iObserve. However, iObserve does not support geo-location processing. So, iObserve needs to be extended. This extension is made directly in the original iObserve and Kieker. Upon detected geo-location or deployment changes, the runtime model is updated and the next filter stage is invoked. We have determined the required data for a privacy analysis and provide a suited transformation of these information to the PCM Privacy runtime model.

The compliance checker, mostly referenced as \textit{Privacy Analysis}, represents the analysis phase in the MAPE loop. It analyses the current runtime model for privacy violations. The fundamental principles were discussed in \autoref{ch:PrivacyConcept}. When a privacy violation is detected, the MAPE Planning phase gets activated.

The planning stages task is to find a privacy compliant redeployment model. For this job PerOpteryx is used. PerOpteryx (\autoref{sec:Foundations:peropteryx}) is a complex model generation and optimization framework. However, PerOpteryx doesn't support privacy or deployment constraints and therefore needs an extension. Furthermore, an output model needs to be selected as the final redeployment candidate, which is transmitted to the final pipeline filter stage.

The execute phase of the MAPE loop compares the architectural re-deployment model to the architectural runtime model. Based on comparison, a migration adaptation sequence is calculated. The adaptation sequence is technology independent and consists of individual adaptation actions. Finally, the adaptation sequence is executed. The execution is based on scripts, that represent a technology dependent implementation of the individual adaptation actions. After the execution the observed software system needs to be in privacy compliant state.