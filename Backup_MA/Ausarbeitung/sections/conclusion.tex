%% LaTeX2e class for student theses
%% sections/conclusion.tex
%% 
%% Karlsruhe Institute of Technology
%% Institute for Program Structures and Data Organization
%% Chair for Software Design and Quality (SDQ)
%%
%% Dr.-Ing. Erik Burger
%% burger@kit.edu
%%
%% Version 1.3, 2016-12-29

\chapter{Conclusion}
\label{ch:Conclusion}

\section{Limitations}
\label{sec:Conculsion:limits}

A cloud provider, like any other person or company, hast own objectives. In the most cases a profit maximization can be assumed as the primary goal. This can be interpreted in multiple ways, from law incompliant behaviour, SLA violations to premium prices for premium services. Nevertheless, in general the assumption stands, that Cloud Providers want to stay SLA and law compliant to avoid lawsuits and reputation loss. Based on this, we assume our providers to be law compliant. 

Furthermore, we state that our providers don’t have any own objectives. This means we don’t have to fear any activates from the provider except providing his cloud services. As a result, we can deploy multiple Type 1 Data (\autoref{sec:PrivacyConcept:dataprivacylevel}) onto one data-centre, but on different (virtual) server, without considering geo-location constraints.

If these assumptions wouldn’t be made, major implications would follow. First, immense security measures must be taken to encrypting any data transmitted, saved or computed by cloud services. The system would have to consider running on an "evil" machine with its own objectives. This is a hot research topic around the globe and not part of the goals tackled by this thesis. Secondly, even if the provider is law and SLA compliant, it is not possible, with the resources at hand, to consider all privacy laws for each country individually. We argue, that these assumptions are a fair trade-off between the reality of a reputable cloud provider and still gaining meaningful results.
\todo{Add cloud provider reveals geo-location via API}

Modern software systems and distributed cloud applications in particular, are designed after separation of concern principle. Systems designed after this principle encapsulate cohesive functionality in a component. If a system ignores this basic design principle, it is possible that our approach does not detect a privacy violation. Since a component gets its data privacy level from the Assembly Connector, a component that saves personal data, but does not communicate them via an Assembly Connector can receive a false privacy level. An example is a component that receives personal information via an user interface and saves or processes them itself. We argue that such a monolithic system stands contrary to the fundamental idea of distributed cloud systems and will be therefore ignored in this further thesis.