%% LaTeX2e class for student theses
%% sections/abstract_en.tex
%% 
%% Karlsruhe Institute of Technology
%% Institute for Program Structures and Data Organization
%% Chair for Software Design and Quality (SDQ)
%%
%% Dr.-Ing. Erik Burger
%% burger@kit.edu
%%
%% Version 1.3, 2016-12-29

\Abstract

With the steadily increasing number of (distributed) cloud systems and more strict data protection regulations, an increasing interest in privacy law compliant cloud applications arises. The major factor to privacy compliance is the distribution of personal data among the geo-location. We developed, implemented and evaluated a privacy analyser for this factor. Further, we extend iObserve after the MAPE feedback loop for automated privacy violation detection, alternative deployment generation and an according cloud system adaptation. This way we can provide continuous privacy compliance on a software architecture level, without code analysis. However, we require the closed world assumption for privacy compliance. PerOpteryx is used for the generation of an alternative, privacy compliant system deployment. Based on this alternative we compute a series of adaptation steps to re-establish privacy compliance. On error occurrence, we make use of the operator-in-the-loop approach of iObserve to help with the system evolution. iObserve and PerOpteryx use the Palladio Component model as Architecture Description Language. In this thesis, we are describing our concepts, point out implementation details and evaluate the iObserve extension. The accuracy evaluation shows our system works as intended and the scalability evaluation reveals the good performance characteristics.